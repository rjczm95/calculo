% Options for packages loaded elsewhere
\PassOptionsToPackage{unicode}{hyperref}
\PassOptionsToPackage{hyphens}{url}
%
\documentclass[
  ignorenonframetext,
]{beamer}
\usepackage{pgfpages}
\setbeamertemplate{caption}[numbered]
\setbeamertemplate{caption label separator}{: }
\setbeamercolor{caption name}{fg=normal text.fg}
\beamertemplatenavigationsymbolsempty
% Prevent slide breaks in the middle of a paragraph
\widowpenalties 1 10000
\raggedbottom
\setbeamertemplate{part page}{
  \centering
  \begin{beamercolorbox}[sep=16pt,center]{part title}
    \usebeamerfont{part title}\insertpart\par
  \end{beamercolorbox}
}
\setbeamertemplate{section page}{
  \centering
  \begin{beamercolorbox}[sep=12pt,center]{part title}
    \usebeamerfont{section title}\insertsection\par
  \end{beamercolorbox}
}
\setbeamertemplate{subsection page}{
  \centering
  \begin{beamercolorbox}[sep=8pt,center]{part title}
    \usebeamerfont{subsection title}\insertsubsection\par
  \end{beamercolorbox}
}
\AtBeginPart{
  \frame{\partpage}
}
\AtBeginSection{
  \ifbibliography
  \else
    \frame{\sectionpage}
  \fi
}
\AtBeginSubsection{
  \frame{\subsectionpage}
}
\usepackage{lmodern}
\usepackage{amssymb,amsmath}
\usepackage{ifxetex,ifluatex}
\ifnum 0\ifxetex 1\fi\ifluatex 1\fi=0 % if pdftex
  \usepackage[T1]{fontenc}
  \usepackage[utf8]{inputenc}
  \usepackage{textcomp} % provide euro and other symbols
\else % if luatex or xetex
  \usepackage{unicode-math}
  \defaultfontfeatures{Scale=MatchLowercase}
  \defaultfontfeatures[\rmfamily]{Ligatures=TeX,Scale=1}
\fi
% Use upquote if available, for straight quotes in verbatim environments
\IfFileExists{upquote.sty}{\usepackage{upquote}}{}
\IfFileExists{microtype.sty}{% use microtype if available
  \usepackage[]{microtype}
  \UseMicrotypeSet[protrusion]{basicmath} % disable protrusion for tt fonts
}{}
\makeatletter
\@ifundefined{KOMAClassName}{% if non-KOMA class
  \IfFileExists{parskip.sty}{%
    \usepackage{parskip}
  }{% else
    \setlength{\parindent}{0pt}
    \setlength{\parskip}{6pt plus 2pt minus 1pt}}
}{% if KOMA class
  \KOMAoptions{parskip=half}}
\makeatother
\usepackage{xcolor}
\IfFileExists{xurl.sty}{\usepackage{xurl}}{} % add URL line breaks if available
\IfFileExists{bookmark.sty}{\usepackage{bookmark}}{\usepackage{hyperref}}
\hypersetup{
  pdftitle={Tema 1 - Números reales},
  pdfauthor={Juan Gabriel Gomila, Arnau Mir y Llorenç Valverde},
  hidelinks,
  pdfcreator={LaTeX via pandoc}}
\urlstyle{same} % disable monospaced font for URLs
\newif\ifbibliography
\usepackage{longtable,booktabs}
\usepackage{caption}
% Make caption package work with longtable
\makeatletter
\def\fnum@table{\tablename~\thetable}
\makeatother
\usepackage{graphicx,grffile}
\makeatletter
\def\maxwidth{\ifdim\Gin@nat@width>\linewidth\linewidth\else\Gin@nat@width\fi}
\def\maxheight{\ifdim\Gin@nat@height>\textheight\textheight\else\Gin@nat@height\fi}
\makeatother
% Scale images if necessary, so that they will not overflow the page
% margins by default, and it is still possible to overwrite the defaults
% using explicit options in \includegraphics[width, height, ...]{}
\setkeys{Gin}{width=\maxwidth,height=\maxheight,keepaspectratio}
% Set default figure placement to htbp
\makeatletter
\def\fps@figure{htbp}
\makeatother
\setlength{\emergencystretch}{3em} % prevent overfull lines
\providecommand{\tightlist}{%
  \setlength{\itemsep}{0pt}\setlength{\parskip}{0pt}}
\setcounter{secnumdepth}{-\maxdimen} % remove section numbering

\title{Tema 1 - Números reales}
\author{Juan Gabriel Gomila, Arnau Mir y Llorenç Valverde}
\date{}

\begin{document}
\frame{\titlepage}

\hypertarget{introducciuxf3n}{%
\section{Introducción}\label{introducciuxf3n}}

\begin{frame}{Cálculo: Las Matemáticas del cambio}
\protect\hypertarget{cuxe1lculo-las-matemuxe1ticas-del-cambio}{}

El término Cálculo identifica un conjunto de instrumentos matemáticos
para efectuar medidas.

El Cálculo Diferencial trata de la medida de tasas de variación: objetos
en movimiento, crecimiento de seres vivos, transmisión de calor, campos
electromagnéticos y un largo etcétera.

Por su parte, el Cálculo Integral, trata de medidas de longitudes,
áreas, volúmenes, flujos de fluidos, \(\ldots\)

Los números reales, \(\mathbb{R}\), proporcionan la base para expresar
los resultados de estas medidas. Por ello empezaremos describiendo estos
números, su estructura y propiedades más significativas.

\end{frame}

\begin{frame}{Números y medida: El conjunto \(\mathbb{N}\) de los
números naturales}
\protect\hypertarget{nuxfameros-y-medida-el-conjunto-mathbbn-de-los-nuxfameros-naturales}{}

Los números naturales \(\mathbb{N} = \{1,2,3, \ldots ,n, \ldots \}\)

\textbf{Suma de números naturales:}

PC. Propiedad conmutativa: \(m+n =n+m\), para todo
\(m,n \in \mathbb{N}\)

PA. Propiedad asociativa: \(m+(n+p) = (m+n)+p\), para todo
\(m,n,p \in \mathbb{N}\)

\end{frame}

\begin{frame}{Números y medida: El conjunto \(\mathbb{N}\) de los
números naturales}
\protect\hypertarget{nuxfameros-y-medida-el-conjunto-mathbbn-de-los-nuxfameros-naturales-1}{}

\textbf{Producto de números naturales:}

EN. Elemento neutro, \(1\): \(n \times 1 = 1 \times n =n\), para todo
\(n \in \mathbb{N}\).

PC. Propiedad conmutativa: \(n \times m = m \times n\), para todo
\(m,n \in \mathbb{N}\).

PA. Propiedad asociativa:
\(n\times (m \times p) = (n \times m)\times p\), para todo
\(m,n,p \in \mathbb{N}\)

PD. Propiedad distributiva del producto respecto de la suma:
\(m \times (n+p) = m\times n +m\times p\), para todo
\(m,n,p \in \mathbb{N}\).

\end{frame}

\begin{frame}{El conjunto \(\mathbb{N}\) de los números naturales.}
\protect\hypertarget{el-conjunto-mathbbn-de-los-nuxfameros-naturales.}{}

\textbf{Principio de inducción}:

Si un subconjunto \(A\) de números naturales es tal que

\begin{enumerate}
[a)]
\item
  \(1 \in A\), y
\item
  Siempre que \(n \in A\) es \(n+1 \in A\)
\end{enumerate}

entonces \(A = \mathbb{N}\).

\textbf{Principio de la buena ordenación}:

Todo subconjunto de números naturales no vacío tiene primer elemento.

\end{frame}

\begin{frame}{Principio de inducción: \textbf{Ejemplos}}
\protect\hypertarget{principio-de-inducciuxf3n-ejemplos}{}

\textbf{Ejemplos}

\textbf{Ejemplo 1}. Demuestra que para todo \(n\in \mathbb{N}\) se
verifica que \(2^{n-1}\leq n!\).

Efectivamente: para \(n=1\) tenemos que \(1=2^0 \leq 2 =2!\). Supongamos
que es cierto para \(n=k-1\), es decir que \(2^{k-2} \leq (k-1)!\),
entonces
\(2^{k-1} = 2\cdot 2^{k-2} \leq 2 \cdot (k-1)! \leq k\cdot(k-1)! =k!\),
puesto que \(2\leq k\). En definitiva es \(2^{n-1}\leq n!\) para todo
\(n \in \mathbb{N}\)

\textbf{Ejemplo 2}. Demuestra que, para todo \(n \in \mathbb{N}\) se
verifica que \(\displaystyle{\sum_{k=1}^n k = \frac{n(n+1)}{2}}\).

En primer lugar, para \(k=1\) la igualdad se verifica puesto que las dos
expresiones son iguales a \(1\). Supongamos que es cierta hasta \(n-1\)
es decir que \(\displaystyle{\sum_{k=1}^{n-1} k = \frac{(n-1)n}{2}}\),
entonces \[
\sum_{k=1}^n k = \frac{n(n+1)}{2} = n+\sum_{k=1}^{n-1} k = n+  \frac{(n-1)n}{2}= \frac{2n+n^2-n}{2} = \frac{n(n+1)}{2}
\] Por lo tanto la igualdad es cierta para todo \(n \in \mathbb{N}\)

\end{frame}

\begin{frame}{Números y medida: El conjunto \(\mathbb{Z}\) de los
números enteros}
\protect\hypertarget{nuxfameros-y-medida-el-conjunto-mathbbz-de-los-nuxfameros-enteros}{}

Los números enteros
\(\mathbb{Z} = \{ \ldots ,-n, \ldots -3,-2,-1,0,1,2,3, \ldots , n \ldots\}\)

\textbf{Suma:}

EN. Elemento neutro, \(0\): \(a+0=0+a=a\), para todo
\(a \in \mathbb{Z}\)

ES. Elemento simétrico, \(-a\): \(a +(-a) = -a+a = 0\),
\(a \in \mathbb{Z}\).

PC. Propiedad conmutativa: \(a+b =b+a\), para todo
\(a,b \in \mathbb{Z}\)

PA. Propiedad asociativa: \(a+(b+c) = (a+b)+c\), para todo
\(a,b,c \in \mathbb{Z}\)

\end{frame}

\begin{frame}{Números y medida: El conjunto \(\mathbb{Z}\) de los
números enteros}
\protect\hypertarget{nuxfameros-y-medida-el-conjunto-mathbbz-de-los-nuxfameros-enteros-1}{}

\textbf{Producto:}

EN. Elemento neutro, \(1\): \(a \times 1 = 1 \times a =a\), para todo
\(a \in \mathbb{Z}\).

PC. Propiedad conmutativa \(a \times b = b \times a\), para todo
\(a,b \in \mathbb{Z}\).

PA. Propiedad asociativa:
\(a\times (b \times c) = a \times b)\times c\), para todo
\(a,b,c \in \mathbb{Z}\)

PD. Distributiva del producto respecto de la suma:
\(a \times (b+c) = a\times b +a \times c\), para todo
\(a,b,c \in \mathbb{Z}\).

\end{frame}

\begin{frame}{El conjunto \(\mathbb{Z}\) de los números enteros}
\protect\hypertarget{el-conjunto-mathbbz-de-los-nuxfameros-enteros}{}

\textbf{División entera}

Dados dos números enteros \(a\) y \(b\), existen otros dos enteros \(q\)
y \(r\), con \(0 \leq r<q\) tales que

\[
a=bq+r
\] \(q\): cociente, \(r\): resto

Si \(r=0\), entonces decimos que \(a\) es un \textbf{múltiplo} de \(b\)
y tambien que \(b\) es un \textbf{divisor} de \(a\)

Si \(r>0\) entonces \(a\) y \(b\) son \textbf{primos} entre si.

\end{frame}

\begin{frame}{Números y medida: El conjunto \(\mathbb{Q}\) de los
números racionales}
\protect\hypertarget{nuxfameros-y-medida-el-conjunto-mathbbq-de-los-nuxfameros-racionales}{}

Los números racionales:
\(\mathbb{Q} = \left\{ \dfrac{p}{q}: p,q \in \mathbb{Z}, q \neq 0 \right\}\)

\textbf{Suma:}

EN. Elemento neutro, \(0\): \(p+0=0+p=p\), para todo
\(p \in \mathbb{Q}\)

ES. Elemento simétrico, \(-p\): \(p +(-p) = -p+p = 0\),
\(p \in \mathbb{Q}\).

PC. Propiedad conmutativa: \(p+q =q+p\), para todo
\(p,q \in \mathbb{Q}\)

PA. Propiedad asociativa: \(p+(q+r) = (p+q)+r\), para todo
\(p,q,r \in \mathbb{Q}\)

\end{frame}

\begin{frame}{Números y medida: El conjunto \(\mathbb{Q}\) de los
números racionales}
\protect\hypertarget{nuxfameros-y-medida-el-conjunto-mathbbq-de-los-nuxfameros-racionales-1}{}

\textbf{Producto:}

EN. Elemento neutro, \(1\): \(p \cdot 1 = 1 \cdot p =p\), para todo
\(p \in \mathbb{Q}\).

ES. Elemento simétrico, \(\dfrac{1}{p}\).
\(p \cdot \dfrac{1}{p} = \dfrac{1}{p} \cdot p =1\), para todo
\(p \in \mathbb{Q}\), \(p \neq 0\)

PC. Propiedad conmutativa \(p \cdot q = q \cdot p\), para todo
\(p,q \in \mathbb{Q}\).

PA. Propiedad asociativa: \(p\cdot (q \cdot r) = p \cdot q)\cdot r\),
para todo \(p,q,r \in \mathbb{Q}\)

PD. Distributiva del producto respecto de la suma:
\(p \cdot (q+r) = p \cdot q + p \cdot r\), para todo
\(p,q,r \in \mathbb{Q}\).

\end{frame}

\begin{frame}{El conjunto \(\mathbb{Q}\) de los números racionales}
\protect\hypertarget{el-conjunto-mathbbq-de-los-nuxfameros-racionales}{}

Orden: El conjunto de los racionales está ordenado de la forma \[
\dfrac{p}{q} \leq \dfrac{r}{s} \quad \text{ si } \quad  p\cdot  s \leq p \cdot r
\]

Se trata de un orden denso: Dados dos números racionales tales que
\(\dfrac{p}{q} < \dfrac{r}{s}\), entonces existen números racionales
\(\dfrac{m}{n}\) tales que

\[
\dfrac{p}{q} < \dfrac{m}{n} < \dfrac{r}{s}
\]

Es suficiente considerar \(m=p+r\) y \(n=q+s\).

\end{frame}

\begin{frame}{El conjunto \(\mathbb{Q}\) de los números racionales}
\protect\hypertarget{el-conjunto-mathbbq-de-los-nuxfameros-racionales-1}{}

Los números racionales no son suficientes puesto que hay medidas cuyo
resultado no és un número racional:

\begin{itemize}
\tightlist
\item
  La longitud de la diagonal de un cuadrado con el lado como unidad
  (\(\sqrt{2}\))
\item
  La longitud de una circunferencia tomando el diámetro como unidad
  (\(\pi\))
\end{itemize}

\begin{figure}
\includegraphics[width=600px]{Images/sqrt2pi} \end{figure}

\end{frame}

\begin{frame}{La raiz cuadrada de 2 no es un número racional,
\(\sqrt{2} \notin \mathbb{Q}\)}
\protect\hypertarget{la-raiz-cuadrada-de-2-no-es-un-nuxfamero-racional-sqrt2-notin-mathbbq}{}

Proposición

No existe ningún número racional \(q \in \mathbb{Q}\) tal que \(q^2=2\).

\textbf{Demostración:}

Supongamos que sí, que para un cierto
\(q =\dfrac{a}{b} \in \mathbb{Q}\), es \(q^2=\dfrac{a^2}{b^2} = 2\).

Podemos elegir \(a\) y \(b\) de tal manera que \(\\mcd(a,b)=1\)

Entonces seria \(a^2=2b^2\), lo qual significa que \(a\) es un múltiplo
de \(2\).

Dado que se trata de un número primo, debe ser \(a\) un múltiplo de
\(2\), es decir \(a=2k\)

Por lo tanto \(a^2 = 4k^2\), lo que significa que \(b^2=2k^2\) y que
\(b\) tambien debería ser múltiplo de \(2\), lo que contradice el hecho
que \(a\) y \(b\) sean primos entre sí.

\end{frame}

\begin{frame}{Los números reales: \(\mathbb{R}\)}
\protect\hypertarget{los-nuxfameros-reales-mathbbr}{}

Supondremos la existencia de un conjunto \(\mathbb{R}\) de números en el
que hay definidas dos operaciones, suma y producto, con las siguientes
propiedades:

\begin{itemize}
\tightlist
\item
  Propiedades de la suma:

  \begin{itemize}
  \item
    Conmutativa: \(a+b= b+a\), para todos los \(a,b \in \mathbb{R}\)
  \item
    Asociativa: \(a+(b+c) = (a+b)+c\) para todos los
    \(a,b,c \in \mathbb{R}\)
  \item
    Existe un elemento neutro, \(0 \in \mathbb{R}\), para la suma:
    \(a+0=a=0+a\).
  \item
    Para cada \(a \in \mathbb{R}\) existe un elemento
    \(-a \in \mathbb{R}\), simétrico de \(a\), tal que
    \(a+(-a)=0=(-a)+a\)
  \end{itemize}
\end{itemize}

\end{frame}

\begin{frame}{Los números reales: \(\mathbb{R}\)}
\protect\hypertarget{los-nuxfameros-reales-mathbbr-1}{}

\begin{itemize}
\item
  Propiedades del producto de números reales:

  \begin{itemize}
  \item
    Conmutativa: \(a\cdot b=b \cdot a\), para todos los
    \(a,b \in \mathbb{R}\).
  \item
    Asociativa: \(a \cdot (b \cdot c) = (a\cdot b)\cdot c\) para todos
    los \(a,b,c \in \mathbb{R}\)
  \item
    Existe un elemento neutro, \(1 \in \mathbb{R}\), para el producto:
    \(a\cdot1=a=1\cdot a\).
  \item
    Para cada \(a \in \mathbb{R}\), \(a\neq 0\), existe un elemento
    \(\dfrac{1}{a} = a^{-1} \in \mathbb{R}\), inverso de \(a\), tal que
    \(a \cdot a^{-1}=1=a ^{-1}\cdot a\)
  \item
    El producto es distributivo respecto de la suma:
    \(a\cdot (b+c) = a\cdot b + a \cdot c\) para todos los
    \(a,b,c \in \mathbb{R}\).
  \end{itemize}
\end{itemize}

\end{frame}

\begin{frame}{Los números reales incluyen los naturales, enteros y
racionales:
\(\mathbb{N} \subset \mathbb{Z} \subset \mathbb{Q} \subset \mathbb{R}\)}
\protect\hypertarget{los-nuxfameros-reales-incluyen-los-naturales-enteros-y-racionales-mathbbn-subset-mathbbz-subset-mathbbq-subset-mathbbr}{}

\begin{itemize}
\item
  Dado que en \(\mathbb{R}\) existe un elemento neutro para el producto,
  \(1\), podemos asimilar el número natural \(n\) con el número real
  \(1+ \ldots \overset{n)}{} \ldots + 1\), es decir
  \(\mathbb{N} \subset \mathbb{R}\)
\item
  Dado que en \(\mathbb{R}\), existe \(-1\), el simétrico de \(1\),
  podemos asimilar el número entero \(-n\) con el número real
  \((-1)+(-1)+ \ldots \overset{n)}{} \ldots +(-1)\). En definitiva
  \(\mathbb{N} \subset \mathbb{Z} \subset \mathbb{R}\)
\item
  Dado que para cada \(a \in \mathbb{R}, \quad a\neq 0\) existe
  \(\dfrac{1}{a}\), el inverso de \(a\). Entonces para cada
  \(n,m \in \mathbb{Z}\), con \(m \neq 0\), será
  \(\dfrac{n}{m} \in \mathbb{R}\), es decir,
  \(\mathbb{N} \subset \mathbb{Z} \subset \mathbb{Q} \subset \mathbb{R}\).
\end{itemize}

\end{frame}

\begin{frame}{El orden en los números reales: \(\mathbb{R}\)}
\protect\hypertarget{el-orden-en-los-nuxfameros-reales-mathbbr}{}

Para definir una relación de orden entre los números reales, supondremos
que existe un subconjunto \(P\) de números reales tal que:

\[
P \cup  \{ 0 \} \cup (-P) = \mathbb{R} 
\]

donde \(-P = \{-x: x \in \mathbb{R} \}\), tales que
\(P\cap (-P) = \emptyset\).

Es decir, todo número real o está en \(P\) o és igual a \(0\) o su
simétrico está en \(P\), y sólo una de estas tres opciones.

Además, \(P\), satisface las propiedades:

\begin{enumerate}
[(1)]
\item
  \(P+P \subset P\), es decir, la suma de dos números de \(P\) és un
  número de \(P\), y
\item
  \(P \cdot P \subset P\), el producto de dos números de \(P\) también
  es un número de \(P\)
\end{enumerate}

\end{frame}

\begin{frame}{El orden en los números reales}
\protect\hypertarget{el-orden-en-los-nuxfameros-reales}{}

Definición

Dados dos números reales, \(a\) y \(b\), diremos que \(a\) és
\textbf{menor o igual} que \(b\) y escribiremos \(a \leq b\), si
\(b-a \in P \cup \{0\}\).

Es decir, \(a \leq b\) si la diferencia \(b-a\) está en \(P\) o \(a=b\).

Si \(a\neq b\) diremos que \(a\) es estrictamente menor que \(b\) y
escribiremos \(a<b\).

Diremos que \(a \geq b\), es decir que \(a\) es \textbf{mayor o igual}
que \(b\), si \(b \leq a\).

Obsérvese que \(a \in P\) es equivalente a decir que \(a > 0\). De ahí
que a los elementos de \(P\) se les llame \textbf{números positivos}

\end{frame}

\begin{frame}{Propiedades del orden en los números reales}
\protect\hypertarget{propiedades-del-orden-en-los-nuxfameros-reales}{}

Proposición

\begin{enumerate}
\item
  \(a \leq a\) (Propiedad reflexiva),
\item
  Si \(a \leq b\) y \(b \leq a\), entonces \(a=b\) (Propiedad
  antisimétrica)
\item
  Si \(a \leq b\) y \(b \leq c\), entonces \(a \leq c\) (Propiedad
  transitiva)
\end{enumerate}

\textbf{Demostración:}

\begin{enumerate}
\item
  Obviamente \(a-a=0 \in P \cup \{ 0\}\).
\item
  Si \(a-b\) y \(b-a\) estan los dos en \(P \cup \{ 0\}\), dado que un
  número y su simétrico no pueden estar a la vez en \(P\), la única
  posibilidad que queda es que sean iguales, es decir que \(a-b=b-a=0\).
\item
  Si \(a=b\) o \(b=c\), el resultado es inmediato. Si los tres son
  diferentes, entonces si \(b-a\) y \(c-b\),estan en \(P\), su suma
  \((b-a)+(c-b)=c-a\) también está en \(P\), es decir \(a \leq c\).
\end{enumerate}

\end{frame}

\begin{frame}{Más propiedades del orden en los números reales}
\protect\hypertarget{muxe1s-propiedades-del-orden-en-los-nuxfameros-reales}{}

Proposición

\begin{itemize}
\tightlist
\item
  Si \(a \in \mathbb{R}\), entonces \(a^2 \in P \cup \{0\}\),
\item
  \(1 \in P\),
\item
  \(\mathbb{N} \subset P\).
\end{itemize}

\textbf{Demostración:}

\begin{itemize}
\item
  \(a\) puede ser de \(P\), \(0\) o de \(-P\). Si es de \(P\), entonces
  \(a^2 \in P\), por ser el producto de dos elementos de \(P\). Si
  \(a=0\), entonces \(a^2 = 0\). Finalmente, si \(a \in -P\), entonces
  \(-a \in P\), por lo que \((-a)(-a) = (-1)a(-1)a=(-1)(-1)aa = a^2\) es
  decir, \(a^2 \in P\), al ser, nuevamente, el producto de dos elementos
  de \(P\).
\item
  Inmediato a partir del apartado anterior, puesto que \(1=1^2\)
\item
  Como hemos visto \(n = 1+ \ldots \overset{n)}{} \ldots + 1\), es decir
  cualquier número natural es la suma de números positivos y, por lo
  tanto, positivo.
\end{itemize}

\end{frame}

\begin{frame}{Más propiedades del orden en \(\mathbb{R}\)}
\protect\hypertarget{muxe1s-propiedades-del-orden-en-mathbbr}{}

Proposición.

Sean \(a,b, c \in \mathbb{R}\)

\begin{enumerate}
[a)]
\tightlist
\item
  Si \(a>b\), entonces \(a+c > b+c\).
\item
  Si \(a>b\) y \(c>d\), entonces \(a+c >b+d\).
\item
  Si \(a>b\) y \(c>0\), entonces \(ac>bc\).
\item
  Si \(a>b\) y \(c<0\), entonces \(ac<bc\).
\item
  Si \(a>0\), entonces \(\dfrac{1}{a} >0\).
\item
  Si \(a<0\), entonces \(\dfrac{1}{a} <0\)
\end{enumerate}

\end{frame}

\begin{frame}{Todavía más propiedades del orden en \(\mathbb{R}\)}
\protect\hypertarget{todavuxeda-muxe1s-propiedades-del-orden-en-mathbbr}{}

Proposición.

Sean \(a,b \in \mathbb{R}\)

\begin{enumerate}
[a)]
\tightlist
\item
  Si \(a < b\) entonces \(a < \dfrac{a+b}{2} <b\).
\item
  Si \(b>0\), entonces \(0<\dfrac{b}{2} < b\)
\item
  Si \(0\leq a < \epsilon\) para todo número real positivo \(\epsilon\),
  entonces \(a=0\).
\item
  Si \(a-\epsilon < b\) para todo \(\epsilon > 0\), entonces \(a\leq b\)
\end{enumerate}

\end{frame}

\begin{frame}{Valor absoluto}
\protect\hypertarget{valor-absoluto}{}

Definición

Dado un número real \(a\), el \textbf{valor absoluto} de \(a\),
representado por \(|a|\), es el mayor de los dos valores \(\{a,-a \}\).

Así \(|-3|=3\) y \(|45|=45\), es decir \(|a| = a\), si \(a \geq 0\) y,
si \(a <0\), entonces \(|a| = -a\),

Es fácil comprobar que \(|a| = \sqrt{a^2}\).

Un error frecuente que se debe evitar es considerar que
\(\sqrt{a^2}=a\).

\end{frame}

\begin{frame}{Valor absoluto: propiedades}
\protect\hypertarget{valor-absoluto-propiedades}{}

Proposición

\begin{itemize}
\tightlist
\item
  \(|a| \geq 0\) y \(|a|=0\) si, y sólo si, \(a=0\).
\item
  \(|-a|=|a|\), para todo \(a \in \mathbb{R}\).
\item
  \(|ab|= |a||b|\), para toda \(a,b \in \mathbb{R}\).
\item
  Si \(c \geq 0\), entonces \(|a| \leq c\), si y sólo si
  \(-c \leq a \leq c\)
\item
  \(-|a| \leq a \leq |a|\), para todo \(a \in \mathbb{R}\).
\end{itemize}

\end{frame}

\begin{frame}{Desigualdad triangular}
\protect\hypertarget{desigualdad-triangular}{}

Proposición

Para toda \(a,b \in \mathbb{R}\):

\begin{itemize}
\tightlist
\item
  \(|a+b| \leq |a| + |b|\) (Desigualdad triangular).
\item
  \(||a|-|b|| \leq |a-b|\).
\item
  \(|a-b| \leq |a|+|b|\).
\end{itemize}

\textbf{Demostración:}

\begin{itemize}
\tightlist
\item
  Dado que \(-|a| \leq a \leq |a|\) y \(-|b| \leq b \leq |b|\), tenemos
  que
\end{itemize}

\[ 
-(|a|+|b|) \leq a+b \leq  |a| + |b|
\] es decir, \(|a+b| \leq |a| + |b|\).

\begin{itemize}
\tightlist
\item
  \(|a| = |a-b+b| \leq |a-b| + |b|\), es decir \(|a|-|b| \leq |a-b|\).
  Análogamente comprobaríamos que \(|b|-|a| \leq |b-a|\).
\end{itemize}

\end{frame}

\begin{frame}{Intervalos en \(\mathbb{R}\)}
\protect\hypertarget{intervalos-en-mathbbr}{}

Definición

Dados números reales \(a,b \in \mathbb{R}\) tales que \(a<b\), el
\textbf{intervalo cerrado de extremos \(a\) y \(b\)} es el conjunto

\[
[a,b] = \{x\in \mathbb{R}: a \leq x \leq b \}
\] Por su parte, \textbf{intervalo abierto de extremos \(a\) y \(b\)} es
el conjunto:

\[
(a,b) = \{x\in \mathbb{R}: a < x < b \}
\] \textbf{Intervalos semiabiertos:}

\[
(a,b] = \{ x \in \mathbb{R}: a<x \leq b \} \quad \text{ y } \quad [a,b) = \{ x \in \mathbb{R}: a \leq x < b \}
\]

\end{frame}

\begin{frame}{Observación importante sobre el valor absoluto}
\protect\hypertarget{observaciuxf3n-importante-sobre-el-valor-absoluto}{}

A menudo usaremos desigualdades del estilo

\[
|x-a|\leq \epsilon
\]

De acuerdo con la definición de valor absoluto, esta desigualdad
equivale a las dos desigualdades

\[
-\epsilon \leq x-a \leq \epsilon
\]

y también

\[
a-\epsilon \leq x \leq a+ \epsilon
\]

o, lo que es lo mismo,

\[
x \in [a-\epsilon,a+\epsilon]
\]

\end{frame}

\begin{frame}{Entornos abiertos y cerrados}
\protect\hypertarget{entornos-abiertos-y-cerrados}{}

Las consideraciones anteriores llevan a las siguientes

\textbf{Definiciones}

\textbf{Entorno abierto:} Dados un punto \(a \in \mathbb{R}\) y un
\(\epsilon >0\), el \textbf{entorno abierto} de centro \(a\) y radio
\(\epsilon\) es el conjunto de puntos
\(V_{\epsilon}(a)=\{ x \in \mathbb{R}: |x-a|<\epsilon \} = (a-\epsilon, a+\epsilon)\).

\textbf{Entorno cerrado:} Dados un punto \(a \in \mathbb{R}\) y un
\(\epsilon >0\), el \textbf{entorno cerrado} de centro \(a\) y radio
\(\epsilon\) es el conjunto de puntos
\(\overline{V}_{\epsilon}(a)=\{ x \in \mathbb{R}: |x-a| \leq \epsilon \} = [a-\epsilon, a+\epsilon]\).

\end{frame}

\begin{frame}{Entornos abiertos y cerrados}
\protect\hypertarget{entornos-abiertos-y-cerrados-1}{}

\textbf{Definición}

\textbf{Entorno reducido} de un punto: un entorno abierto (cerrado) sin
el punto, es decir \(V^*_{\epsilon}(a)=V_{\epsilon}(a) \setminus \{a\}\)
\(\left(\overline{V}^*_{\epsilon}(a) = \overline{V}_{\epsilon}(a) \setminus \{a\}\right)\)

Vecindad o bola son también nombres habituales para los entornos.

\begin{figure}
\includegraphics[width=700px]{Images/vecindad} \caption{Entorno de un punto}\label{fig:vecin}
\end{figure}

\end{frame}

\begin{frame}{Conjuntos acotados}
\protect\hypertarget{conjuntos-acotados}{}

Definición

\begin{itemize}
\item
  Un subconjunto \(A \subset \mathbb{R}\) está \textbf{acotado
  superiormente} si existe un \(\alpha \in \mathbb{R}\) tal que
  \(a \leq \alpha\) para todo \(a \in A\). Al elemento \(\alpha\) se le
  llama \textbf{cota superior} del conjunto \(A\).
\item
  Un subconjunto \(A \subset \mathbb{R}\) está \textbf{acotado
  inferiormente} si existe un \(\alpha \in \mathbb{R}\) tal que
  \(a \geq \alpha\) para todo \(a \in A\). En este caso \(\alpha\) es
  \textbf{una cota inferior} de \(A\).
\item
  Un subconjunto \(A \subset \mathbb{R}\) está \textbf{acotado} si si
  està acotado superiormente e inferiormente.
\end{itemize}

\end{frame}

\begin{frame}{Conjuntos acotados: \textbf{Ejemplos}}
\protect\hypertarget{conjuntos-acotados-ejemplos}{}

\textbf{Ejemplos}.

\begin{enumerate}
[1)]
\item
  \(1\) es una cota inferior del conjunto de los números naturales.
  Veremos más adelante que este conjunto no está acotado superiormente
  en \(\mathbb{R}\).
\item
  El conjunto \(A =\{q \in \mathbb{Q}\) tal que \(q^2 \leq 2 \}\) está
  acotado superiormente. Tambien lo está inferiormente.
\item
  El conjunto \(\{ y \mathbb{R}: y=x^2+3\}\) esta acotado inferiormente
  por \(3\). No lo está superiormente.
\end{enumerate}

\textbf{Observación}

La existencia de una cota superior supone la existencia de infinitas
cotas, puesto que cualquier número mayor que una cota superior también
será una cota superior. Análogamente para las cotas inferiores:
cualquier número menor que una cota inferior es también una cota
inferior.

\end{frame}

\begin{frame}{Supremos e ínfimos}
\protect\hypertarget{supremos-e-uxednfimos}{}

Definición

\begin{itemize}
\item
  El \textbf{supremo} de un conjunto acotado superiormente es la menor
  de las cotas superiores de dicho conjunto. Un conjunto puede estar
  acotado superiormente, pero no tener supremo (en \(\mathbb(Q)\)).
\item
  El \textbf{ínfimo} de un conjunto acotado inferiormente es la mayor de
  las cotas inferiores de dicho conjunto. Tambien puede pasar que un
  conjunto acotado inferiormente no tenga ínfimo.
\end{itemize}

\textbf{Ejemplos}

\begin{itemize}
\tightlist
\item
  El subconjunto \(A\) de \(\mathbb{Q}\) definido por
  \(A =\{q \in \mathbb{Q}\) tal que \(q^2 \leq 2 \}\), está acotado
  superiormente, pero, como se ha demostrado no tiene supremo, puesto
  que este supremo deberia ser \(\sqrt{2}\) que no es un número
  racional.
\end{itemize}

\end{frame}

\begin{frame}{Propiedades del supremo y del ínfimo}
\protect\hypertarget{propiedades-del-supremo-y-del-uxednfimo}{}

\textbf{Proposición}

El supremo (el ínfimo) de un conjunto, si existe, és único.

\textbf{Demostración:} Supongamos que \(s_1\) y \(s_2\) son dos supremos
del conjunto \(A\). Entonces, dado que \(s_1\) es supremo, es una cota
superior, por lo tanto debe ser mayor o igual que el supremo \(s_2\), es
decir \(s_1 \geq s_2\). Por otro lado, dado que \(s_2\) es supremo, es
cota superior, por lo tanto debe ser mayor o igual que el supremo
\(s_1\), es decir \(s_2 \geq s_1\). En definitiva, debe ser \(s_1=s_2\).

\textbf{Proposición}

Un número \(s\) es el supremo de un subconjunto no vacío \(A\) de
\(\mathbb{R}\) si, y sólo si, \(s\) satisface las condiciones:

\begin{enumerate}
[a)]
\item
  Es cota superior: \(s\geq a\), para todo \(a \in A\).
\item
  Es la menor de las cotas superiores: Si \(b < s\), entonces existe
  \(a' \in A\) tal que \(b<a'\).
\end{enumerate}

\end{frame}

\begin{frame}{Propiedades del supremo y del ínfimo}
\protect\hypertarget{propiedades-del-supremo-y-del-uxednfimo-1}{}

\textbf{Proposición}

Una cota superior \(s\) de un subconjunto no vacío \(A\) de
\(\mathbb{R}\), es el supremo de \(A\), si, y sólo si, para cada
\(\epsilon >0\) existe un \(s_{\epsilon} \in A\), tal que
\(s-\epsilon < s_\epsilon\)

\textbf{Demostración:}

\((\leftarrow)\)Supongamos que \(s\) es una cota superior que satisface
la propiedad enunciada, entonces si \(a<s\), tomando \(\epsilon = s-a\),
entonces \(\epsilon >0\), por lo que existe un número \(s_{\epsilon}\)
tal que \(a=s-\epsilon < s_{\epsilon}\), es decir que \(a\) no es cota
superior de \(A\), es decir \(s=\sup A\).

\((\rightarrow)\) Recíprocamente, supongamos que \(s = \sup A\) y sea
\(\epsilon\) un número positivo. Como \(s-\epsilon <s\), \(s-\epsilon\)
no es cota superior de A, por lo que debe existir \(s_{\epsilon}\) tal
que \(s-\epsilon < s_{\epsilon}\).

\end{frame}

\begin{frame}{Máximos y mínimos}
\protect\hypertarget{muxe1ximos-y-muxednimos}{}

\textbf{Definición}

Si el supremo de un conjunto es un elemento del conjunto, entonces
recibe el nombre de \textbf{máximo}.

Análogamente, si el ínfimo de un conjunto es un elemento del conjunto,
entoces recibe el nombre de \textbf{mínimo}

\textbf{Ejemplos}

\(2\) es un máximo del conjunto \(\{x \in \mathbb{R}: x \leq 2 \}\).

Sin embargo no lo es del conjunto \(\{x \in \mathbb{R}: x < 2 \}\), del
que sigue siendo el supremo.

\end{frame}

\begin{frame}{El axioma del supremo}
\protect\hypertarget{el-axioma-del-supremo}{}

Axioma del supremo

Todo subconjunto de \(\mathbb{R}\) acotado superiormente tiene supremo.

Esta es, efectivamente, una propiedad que no se satisface en
\(\mathbb{Q}\), como se ha demostrado con el conjunto
\(A =\{q \in \mathbb{Q}\) tal que \(q^2 \leq 2 \}\).

Resulta fácil comprobar que si \(s= \sup A\), entonces \(-s= \inf(-A)\),
por lo que el axioma del supremo tambien implica que

Todo subconjunto de \(\mathbb{R}\) acotado inferiormente tiene ínfimo.

\end{frame}

\begin{frame}{Consecuencias del axioma del supremo.}
\protect\hypertarget{consecuencias-del-axioma-del-supremo.}{}

\textbf{Proposición. (Propiedad arquimediana)}

El conjunto \(\mathbb{N}\) de los números naturales no está acotado
superiormente en \(\mathbb{R}\).

\textbf{Demostración:}

Supongamos que existen cotas superiores para \(\mathbb{N}\) en
\(\mathbb{R}\), por el axioma del supremo, existe
\(\alpha = \sup \mathbb{N}\). Al ser \(\alpha\) la menor de las cotas
superiores, resulta que \(\alpha -1\) no es cota superior. Por lo tanto
existe un \(n \in \mathbb{N}\) tal que \(\alpha -1 <n\), es decir que
\(\alpha < n+1\), lo cual es absurdo, puesto que habíamos supuesto que
\(\alpha\) era mayor que todos los nombres naturales.

\end{frame}

\begin{frame}{Consecuencias del axioma del supremo.}
\protect\hypertarget{consecuencias-del-axioma-del-supremo.-1}{}

\textbf{Corolario}

Sean \(y,z\) reales positivos. Entonces:

\begin{itemize}
\tightlist
\item
  Existe \(n \in \mathbb{N}\) tal que \(z<ny\).
\item
  Existe \(n \in \mathbb{N}\) tal que \(0 <\dfrac{1}{n}<y\).
\item
  Existe \(n \in \mathbb{N}\) tal que \(n-1 <z < n\).
\end{itemize}

\end{frame}

\begin{frame}{Consecuencias del axioma del supremo: Demostración.}
\protect\hypertarget{consecuencias-del-axioma-del-supremo-demostraciuxf3n.}{}

\textbf{Demostración:}

\begin{itemize}
\item
  Considere \(\dfrac{z}{y}\), se trata de un número positivo, por
  consiguiente existe \(n \in \mathbb{N}\) tal que \(\dfrac{z}{y} <n\),
  es decir \(z < ny\).
\item
  Inmediata a partir del resultado anterior con \(z=1\).
\item
  El conjunto \(\{ m \in \mathbb{N}: z<m \}\) es no vacío. Sea \(n\) el
  primer elemento de este conjunto, entonces
\end{itemize}

\[
n-1 \leq z < n
\]

\end{frame}

\begin{frame}{Consecuencias del Ax. del supremo:
\(\sqrt{2} \in \mathbb{R}\).}
\protect\hypertarget{consecuencias-del-ax.-del-supremo-sqrt2-in-mathbbr.}{}

\textbf{Proposición}

Existe un número real \(p \in \mathbb{R}\) tal que \(p^2 = 2\).

\textbf{Demostración:}

Consideremos el conjunto \(M = \{r \in \mathbb{R}: r^2 <2 \}\). \(M\) es
un subconjunto acotado de \(\mathbb{R}\) (¿Por qué?). Sea
\(p = \sup M\). Demostraremos que \(p^2 =2\).

Supongamos, en primer lugar, que \(p^2<2\), veamos que existe
\(\epsilon >0\), tal que \((p+ \epsilon)^2 <2\). Con este fin,
supongamos que, efectivamente, existe un tal \(\epsilon\), entonces
seria \(p^2+2p \epsilon +{\epsilon}^2 <2\) y, por lo tanto debería ser

\[
\epsilon < \dfrac{2-p^2}{2p+\epsilon}
\]

Consideremos ahora
\(\epsilon = \min \left\{1,\dfrac{2-p^2}{2p+1} \right\}\), entonces
tenemos que
\(\epsilon < \dfrac{2-p^2}{2p+1} < \dfrac{2-p^2}{2p+\epsilon}\), por lo
tanto es \((p+\epsilon)^2 < 2\), es decir \((p+\epsilon) \in M\) y como
\(\epsilon >0\), esto contradice que \(p= \sup M\).

\end{frame}

\begin{frame}{\(\sqrt{2} \in \mathbb{R}\) (Continuación)}
\protect\hypertarget{sqrt2-in-mathbbr-continuaciuxf3n}{}

\textbf{Demostración (Continuación):}

Supongamos ahora que \(p^2>2\). Veremos que existe \(\epsilon >0\) tal
que \((p-\epsilon)^2 >2\), por lo que \(p\) tampoco podrá ser el supremo
de \(M\). Ahora, si \((p-\epsilon)^2 >2\), debe ser
\(p^2-2p\epsilon+\epsilon^2>2\), es decir,
\(p^2-2 > \epsilon(2p -\epsilon)\), es decir \[
\epsilon < \dfrac{p^2-2}{2p-\epsilon}
\] Si ahora consideramos \(\epsilon = \frac{p^2 -2}{2p}\), está claro
que para este \(\epsilon >0\) es
\(\epsilon = \dfrac{p^2 -2}{2p}> \dfrac{p^2-2}{2p-\epsilon}\), es decir
que \((p-\epsilon)^2 >2\), y por lo tanto \(p\) tampoco puede ser el
supremo de \(M\) si \(p^2 >2\).

En definitiva, sólo queda la opción que el supremo \(p\) del conjunto
\(M\), sea tal que \(p^2=2\), es decir que \(\sqrt{2}\) és un número
real.

\textbf{Corolario}

Para todo número real positivo \(b\), existe un número real positivo
\(\beta\) tal que \(\beta^2 =b\).

\end{frame}

\begin{frame}{Los números irracionales (1)}
\protect\hypertarget{los-nuxfameros-irracionales-1}{}

Acabamos de demostrar que hay números en \(\mathbb{R}\) que no estan
\(\mathbb{Q}\), como seria el caso de \(\sqrt{2}\). Por lo tanto el
conjunto de los números reales que no son racionales,
\(\mathbb{R} \setminus \mathbb{Q}\), és un conjunto no vacio. Es el
conjunto de los números irracionales .

Este nombre indica que estos números no se pueden expresar como el
cociente -la razón- de dos enteros.

Conviene observar que ni la suma ni el producto de dos números
irracionales tienen porqué ser números irracionales. (Por ejemplo
\(\sqrt{2} \cdot \sqrt{2} = 2\))

\end{frame}

\begin{frame}{Los números irracionales (2)}
\protect\hypertarget{los-nuxfameros-irracionales-2}{}

Hemos visto que todo número real está entre dos enteros, es decir, dado
un \(r \in \mathbb{N}\), existe un \(n \in \mathbb{Z}\) tal que
\(n \leq r < n+1\). Se puede demostrar que dado un número irracional
existe una representación decimal de dicho número, es decir, existen
\(b_1,b_2, \ldots, b_n, \ldots\), con no todos igual a \(9\), tales que
\[
r = n.b_1b_2b_3 \ldots b_n \ldots
\]

Es importante observar que, la representación decimal de los números
irracionales constará siempre de un número ilimitado de cifras que no se
repiten de forma periódica, puesto que como es sabido, una
representación decimal con un número limitado de cifras o que se repitan
periódicamente siempre corresponde a un número racional.

\end{frame}

\begin{frame}{Más consecuencias del axioma del supremo.}
\protect\hypertarget{muxe1s-consecuencias-del-axioma-del-supremo.}{}

\textbf{Proposición. \(\mathbb{Q}\) es denso en \(\mathbb{R}\)}

Sean \(x,y \in \mathbb{R}\) tales que \(x<y\), entonces existe
\(q \in \mathbb{Q}\) tal que \(x<q<y\). Es decir, entre dos números
reales siempre hay un número racional.

\textbf{Demostración:}

Supongamos que \(x>0\), entonces \(y-x >0\) y, por la propiedad
arquimediana, existe \(n \in \mathbb{N}\) tal que \(\dfrac{1}{n} <y-x\),
es decir que \(ny-nx >1\).

Dado que \(nx >0\), existe \(m \in \mathbb{N}\) tal que \(m-1 <nx<m\).

\(m\) también verifica \(m<ny\), ya que \(m< nx+1<ny\). En definitiva es
\(nx<m<ny\) y, por lo tanto

\[
x<\dfrac{m}{n}<y
\] es decir, \(q = \dfrac{m}{n}\) es el número racional buscado.
Razonamientos análogos sirven para el caso \(x<0\).

\end{frame}

\begin{frame}{Más consecuencias del axioma del supremo.}
\protect\hypertarget{muxe1s-consecuencias-del-axioma-del-supremo.-1}{}

\textbf{Corolario}

Si \(x,y \in \mathbb{R}\), con \(x<y\), entonces existe un número
irracional z (es decir que \(z \in \mathbb{R} \setminus \mathbb{Q}\))
tal que \(x<z<y\). Es decir, entre dos numeros reales, siempre hay un
número irracional.

\textbf{Demostración:}

Consideremos \(s \in \mathbb{R} \setminus \mathbb{Q}\) (por ejemplo
\(\sqrt{2}\)), \(s>0\), entonces, sabemos que existe un racional
\(q \in \mathbb{Q}\) tal que \(\dfrac{x}{s} <q < \dfrac{y}{s}\). Ahora
\(qs \in \mathbb{R} \setminus \mathbb{Q}\), puesto que si
\(z=qs \in \mathbb{Q}\), \(s=\dfrac{z}{q}\) seria el cociente de dos
racionales y, por la tanto racional. En definitiva es \(x<z<y\), con
\(z\) irracional.

\end{frame}

\begin{frame}{Cardinal de un conjunto: Conjuntos infinitos.}
\protect\hypertarget{cardinal-de-un-conjunto-conjuntos-infinitos.}{}

Esta sección está destinada a comparar el ``tamaño'' de los conjuntos de
números \(\mathbb{N}\), \(\mathbb{Z}\), \(\mathbb{Q}\) y \(\mathbb{R}\).
Veremos que, en tanto que los tres primeros tienen el mismo tamaño, en
un sentido que se definirá, el de los números reales es mayor, es decir
que hay muchos más números irracionales que racionales. Cantor fue el
primero que lo descubrió, la cual cosa le supuso no pocos problemas ante
la incomprensión de una buena parte de los matemáticos contemporáneos.

\end{frame}

\begin{frame}{Aplicaciones}
\protect\hypertarget{aplicaciones}{}

Para lo que sigue, es conviente recordar que una \textbf{aplicación},
\(f\) de un conjunto \(A\) en otro \(B\), representada por
\(f: A \rightarrow B\), es una asignación de elementos de \(B\) a
elementos de \(A\), de tal manera a que cada elemento de \(A\) le
corresponde uno, y sólo uno, elemento de \(B\).

Escribiremos \(b=f(a)\), y diremos que \(b\) \emph{es la imagen de}
\(a\) por \(f\), para indicar que \(b\) es un elemento asignado a \(a\).

\end{frame}

\begin{frame}{Cardinal de un conjunto: Conjuntos infinitos.}
\protect\hypertarget{cardinal-de-un-conjunto-conjuntos-infinitos.-1}{}

Una aplicación, \(f\), es \textbf{inyectiva} si a elementos distintos de
\(A\) le corresponden elementos diferentes de \(B\), es decir si
\(a_1 \neq a_2\) entonces \(f(a_1) \neq f(a_2)\), para todo \(a_1\) y
\(a_2\) de \(A\).

Una aplicación es \textbf{exhaustiva} si todos los elementos de \(B\)
son la imagen de algún elemento de \(A\), es decir si para todo
\(b \in B\), existe un \(a \in A\) tal que \(b=f(a)\).

Finalmente, una aplicación es \textbf{biyectiva} si es inyectiva y
exhaustiva a la vez. A menudo, una aplicación biyectiva también es
llamada \textbf{correspondencia uno a uno}.

\textbf{Ejemplo de aplicación}

\(f: \mathbb{N} \rightarrow \mathbb{N}\) definida por \(f(n) = 2n\) es
una aplicación del conjunto de los números naturales en sí mismo. Por
otro lado, la misma asignación entre \(\mathbb{Z}\) y \(\mathbb{N}\), no
es una aplicación, puesto que el doble de un número negativo no es un
número natural.

\end{frame}

\begin{frame}{Ejemplos de aplicaciones}
\protect\hypertarget{ejemplos-de-aplicaciones}{}

\textbf{Ejemplo de aplicación inyectiva}

\textbf{Ejemplo 1. }Sean \(n \in \mathbb{N}\) y \(N_n\) el conjunto de
los \(n\) primeros números naturales. La aplicación \(f(k)=k\) de
\(N_n \rightarrow \mathbb{N}\) es inyectiva, pero no exhaustiva, ya que
cualquier natural mayor que \(n\) no es imagen de ningún elemento de
\(N_n\).

\textbf{Ejemplo 2.}

\begin{figure}
\includegraphics[width=500px]{Images/inyectiva} \end{figure}

Es inyectiva porque no hay dos elementos de \(\cal{B}\) que sean imagen
del mismo elemento de \(\cal{A}\). No es exhaustiva porque hay un
elemento de \(\cal{B}\) que no es imagen de ninguno de \(\cal{A}\).

\end{frame}

\begin{frame}{Ejemplos de aplicaciones}
\protect\hypertarget{ejemplos-de-aplicaciones-1}{}

\textbf{Ejemplo de aplicación exhaustiva}

La aplicación \(h: \mathbb{Z} \rightarrow \mathbb{N}\), definida por
\(h(n)=|n|\), es una aplicación exhaustiva, pero no inyectiva, puesto
que \(h(n)=h(-n)\).

\textbf{Otro ejemplo} es el que viene representado por el esquema:

\begin{figure}
\includegraphics[width=500px]{Images/exhaustiva} \end{figure}

Se trata de una aplicación exhaustiva porque todos los elementos de
\(\cal{B}\) son imagen de alguno de \(\cal{A}\), pero hay dos pares de
elementos de \(\cal{A}\) que tienen la misma imagen.

\end{frame}

\begin{frame}{Ejemplos de aplicaciones}
\protect\hypertarget{ejemplos-de-aplicaciones-2}{}

\textbf{Ejemplo de aplicación biyectiva}

La aplicación de \(f: \mathbb{N} \rightarrow \mathbb{Z}\) definida por
\[
f(n) =
\begin{cases}
\dfrac{n}{2}, &  \text{ si n es par}\\
-\dfrac{n+1}{2},&  \text{ si n es impar}
\end{cases}
\] es una aplicación biyectiva.

\begin{figure}
\includegraphics[width=500px]{Images/biyectiva} \end{figure}

\end{frame}

\begin{frame}{Cardinal de un conjunto: Conjuntos infinitos.}
\protect\hypertarget{cardinal-de-un-conjunto-conjuntos-infinitos.-2}{}

\textbf{Definición: Conjunto finito}

Sea \(N_n\) el conjunto de los \(n\) primeros números naturales. Diremos
que un conjunto \(A\) tiene \(n\) elementos si existe una
correspondencia uno a uno entre \(A\) y \(N_n\).

Un conjunto \textbf{finito} es un conjunto con \(n\) elementos. Es
decir, un conjunto \(A\) es finito si existe una aplicación biyectiva de
\(A\) en un subconjunto de números naturales de la forma \(N_n\).

Un conjunto es \textbf{infinito} si no es finito.

\textbf{Definición. Cardinal de un conjunto}

Dos conjuntos \(A\) y \(B\) tienen el mismo \textbf{cardinal} si existe
una correspondencia uno a uno entre ellos.

\end{frame}

\begin{frame}{\(\mathbb{N}\) es infinito.}
\protect\hypertarget{mathbbn-es-infinito.}{}

\textbf{Proposición}

El conjunto de los números naturales, \(\mathbb{N}\), es infinito.

\textbf{Demostración}

Es suficiente demostrar que para todo \(m \in \mathbb{N}\) no puede
existir una aplicación inyectiva de \(\mathbb{N}\) en \(N_m\), puesto
que, de ser así, no podrá existir una aplicación biyectiva entre esos
dos conjuntos y, por lo tanto \(\mathbb{N}\) no será un conjunto finito.

Ahora, en primer lugar, cada subconjunto de \(\mathbb{N}\) de \(m\)
elementos se puede poner en correspondencia uno a uno com \(N_m\), por
lo tanto será suficiente considerar el propio
\(N_m = \{1,2, \ldots, m\}\) como subconjunto de \(\mathbb{N}\). Una vez
que hemos asignado la imagen de estos primeros \(m\) elementos de
\(\mathbb{N}\) hemos agotado los elementos de \(N_m\), por consiguiente,
para asignar una imagen a \(m+1 \in \mathbb{N}\) no quedará más remedio
que tomar alguno de los elementos de \(N_m\) y, por lo tanto tendríamos
dos naturales distintos con la misma imagen, lo que significa que la
aplicación no puede ser inyectiva.

La técnica aplicada en la demostración anterior, se conoce con el nombre
del \textbf{Principio del palomar} y resulta ser extremadamente útil en
la resolución de diversos tipos de problemas.

\end{frame}

\begin{frame}{Conjuntos numerables.}
\protect\hypertarget{conjuntos-numerables.}{}

Está claro que dos conjuntos finitos con el mismo número de elementos
tienen el mismo cardinal, pero ¿qué pasa con los conjuntos infinitos?

Hemos visto un ejemplo de una aplicación biyectiva entre los conjuntos
\(\mathbb{Z}\) y \(\mathbb{N}\) y, por lo tanto tienen el mismo
cardinal, aunque \(\mathbb{N} \subset \mathbb{Z}\). Conviene observar
que esta es una característica de los conjuntos infinitos: pueden tener
el mismo cardinal que alguno de sus subconjuntos propios.

\textbf{Definición: Conjunto numerable}

Diremos que un conjunto es \textbf{numerable} si tiene el mismo cardinal
que \(\mathbb{N}\). Diremos que un conjunto es \textbf{contable} si es
finito o numerable.

\end{frame}

\begin{frame}{Conjuntos contables}
\protect\hypertarget{conjuntos-contables}{}

\textbf{Proposición}

Si \(A\) es contable y \(B \subset A\), entonces \(B\) es contable.

\textbf{Demostración}

La aplicación \(i: B \rightarrow A\) que envia cada elemento a si mismo
es inyectiva, por lo tanto el cardinal de \(B\) es a lo sumo el de
\(A\).

\textbf{Ejemplos}

1.El conjunto \(\mathbb{Z}\) de los números enteros es numerable.

\begin{enumerate}
\setcounter{enumi}{1}
\item
  También lo es el conjunto de los números naturales pares y, más en
  general, el de los múltiplos de un número natural, \(k\).
\item
  Si \(A\) es un conjunto numerable y \(f:A \rightarrow B\) es una
  aplicación, entonces \(f(A)\) es numerable, puesto que \(f(A)\) tiene
  a lo sumo (caso que f sea inyectiva) tantos elementos como \(A\)
\end{enumerate}

\end{frame}

\begin{frame}{\(\mathbb{Q}\) es numerable}
\protect\hypertarget{mathbbq-es-numerable}{}

\textbf{Proposición (Cantor)}

El conjunto \(\mathbb{Q}\) es numerable.

\textbf{Demostración}

El resultado se sigue del hecho que la aplicación
\(f: \mathbb{Q} \rightarrow \mathbb{N}\) definida por \[
f \left(\dfrac{m}{n} \right) = \dfrac{1}{2}(m+n-2)(m+n-1)+m
\] es biyectiva.

A continuación se muestra una ``demostración'' visual de que
\(\mathbb{Q}\) es numerable.

\end{frame}

\begin{frame}{\(\mathbb{Q}\) es numerable}
\protect\hypertarget{mathbbq-es-numerable-1}{}

El gráfico muestra cómo disponer los elementos de \(\mathbb{Q}\) para
numerarlos.

\begin{figure}
\includegraphics[width=600px]{Images/cantor} \end{figure}

\end{frame}

\begin{frame}{\(\mathbb{R}\) no es numerable}
\protect\hypertarget{mathbbr-no-es-numerable}{}

\textbf{Proposición}

El conjunto de los números reales, \(\mathbb{R}\), no es numerable.

\textbf{Demostración}

Demostraremos que el intervalo abierto \((0,1) \subset \mathbb{R}\) no
es numerable y, por lo tanto, el conjunto \(\mathbb{R}\) no podrá ser
numerable. Supongamos que si, que \((0,1)\) es numerable. Es decir que
podemos disponer estos números, en su representación decimal, de manera
que uno sea el primero, otro el segundo, y así sucesivamente: \[
a_1 = 0.a_{11}a_{12}a_{13} \ldots a_{1n} \ldots \\
a_2 = 0.a_{21}a_{22}a_{23} \ldots a_{2n} \ldots \\
a_3 = 0.a_{31}a_{32}a_{33} \ldots a_{3n} \ldots \\
\ldots\\
a_m = 0.a_{m1}a_{m2}a_{m3} \ldots a_{mn} \ldots\\
\ldots
\] Consideremos ahora el número \(a=0.b_1b_2b_3 \ldots b_n \ldots\), en
el que cada \(b_k=a_{kk}+1\), si \(a_{kk} \neq 9\), y \(b_k = 0\), si
\(a_{kk} = 9\). El número \(a\) no está en la lista de números anterior,
puesto que si ocupase el lugar \(k\) diferiria del correspondiente
\(a_k\) precisamente en la cifra que ocupa el lugar \(k\).

\end{frame}

\begin{frame}{El conjunto de los números irracionales no es numerable}
\protect\hypertarget{el-conjunto-de-los-nuxfameros-irracionales-no-es-numerable}{}

\textbf{Proposición}

La reunión de dos conjuntos finitos es un conjunto finito.

La reunión de dos conjuntos numerables es numerable.

La reunión de dos conjuntos contables es contable.

\textbf{Proposición}

El conjunto de los numeros irracionales no es numerable.

\textbf{Demostración}

Dado que
\(\mathbb{R} = \mathbb{Q} \cup (\mathbb{R} \setminus \mathbb{Q})\), se
sigue que \(\mathbb{R} \setminus \mathbb{Q}\) no puede ser numerable,
puesto que si lo fuera, dado que \(\mathbb{Q}\) es numerable, también lo
seria \(\mathbb{R}\), y ya hemos visto que no es así.

\end{frame}

\begin{frame}{Intervalos anidados (1)}
\protect\hypertarget{intervalos-anidados-1}{}

\textbf{Definición: Sucesión de intervalos anidados}

Una sucesión de \textbf{intervalos anidados} es un conjunto de
intervalos cerrados y acotados de \(\mathbb{R}\), \(I_n = [a_n,b_n]\),
tal que para todo \(n \in \mathbb{N}\), es \(I_{n+1} \supset I_n\) o, lo
que es equivalente, tal que \(a_n \leq a_{n+1} \leq b_{n+1} \leq b_n\)

\textbf{Observación}

Como se verá más adelante, un conjunto indexado por los números
naturales, como lo son \(\{I_n\}\), \(\{a_n\}\) y \(\{b_n\}\), recibe el
nombre de \textbf{sucesión}

\end{frame}

\begin{frame}{Intervalos anidados (2)}
\protect\hypertarget{intervalos-anidados-2}{}

\textbf{Proposición.}

Sea \(\{I_n\}_{n \in \mathbb{N}}\) una sucesión de intervalos anidados
de \(\mathbb{R}\). Entonces existe un \(\xi \in \mathbb{R}\) tal que
\(\xi \in I_n\) para todo \(n \in \mathbb{N}\), o lo que es lo mismo,
tal que \(a_n \leq \xi \leq b_n\), para todo \(n \in \mathbb{N}\)

\textbf{Demostración}

En primer lugar, veamos que \(a_k \leq b_n\) para todo \(k\) y para todo
\(n\): Si \(n \leq k\), entonces \(I_k \subset I_n\) y, por lo tanto,
\(a_k \leq b_k \leq b_n\). Por otra parte, si \(k < n\), entonces
\(I_k \supset I_n\) y, por lo tanto, \(a_k \leq a_n \leq b_n\)

Es decir, el conjunto de extremos inferiores de los intervalos
\(\{a_n: n \in \mathbb{N}\}\) está acotado superiormente por cualquiera
de los \(b_n\). Sea \(\xi\) el supremo de este conjunto. Está claro que
\(a_n \leq \xi\), para todo \(n \in \mathbb{N}\), puesto que \(\xi\) es
el supremo de todos los \(a_n\).

Pero todos los \(b_n\) son cotas superiores de este conjunto, y el
supremo es la menor de las cotas superiores, por lo que
\(\xi \leq b_n\), para todo n.~En definitiva es
\(a_n \leq \xi \leq b_n\), para todo \(n \in \mathbb{N}\) o, lo que es
lo mismo, \(\xi \in I_n\) para todo \(n \in \mathbb{N}\).

\end{frame}

\begin{frame}{Principio de los intervalos anidados.}
\protect\hypertarget{principio-de-los-intervalos-anidados.}{}

\textbf{Proposición: Principio de los intervalos anidados}

Sea \(I_n = [a_n,b_n]\) una sucesión de intervalos anidados tales que
\(\inf\{b_n-a_n: n \in \mathbb{N}\}=0\). Entoces existe un único punto
\(\xi\) que está en todos los intervalos \(I_n\).

\textbf{Demostración}

Hemos visto en la proposición anterior que
\(\xi = \sup\{a_n: n \in \mathbb{N} \}\) está en todos los \(I_n\). Un
razonamiento similar al de la proposición anterior se demuestra que
\(\eta = \inf \{b_n: n \in \mathbb{N} \}\) tambien está en todos los
intervalos. Sólo falta comprobar que \(\xi = \eta\). Ahora bien:

Dado que \(\eta \leq b_n\) y que \(\xi\geq a_n\), para todo \(n\),
tenemos que \(0 \leq \eta - \xi \leq b_n -a_n\), para todo \(n\), por lo
tanto \(0 \leq \eta - \xi \leq \inf\{b_n-a_n: n \in \mathbb{N}\}=0\), es
decir que \[
\xi = \eta
\] Por lo tanto este es el único punto que está en todos los intervalos,
puesto que cualquier punto \(x\) que esté en todos los intervalos es
mayor que todos \(a_n\) y menor que todos los \(b_n\) y, por lo tanto
seria \(\xi \leq x \leq \eta\).

\end{frame}

\begin{frame}{Puntos de acumulación}
\protect\hypertarget{puntos-de-acumulaciuxf3n}{}

Veremos en el apartado de funciones que, a menudo, tenemos una función
definida en todos los puntos próximos a uno dado, pero que no lo está es
ese punto. Tal sería el caso, por ejemplo de la función
\(f(x)= \frac{\sin x}{x}\): no está definida en el punto \(0\), pero si
lo está en todos los puntos de cualquier entorno reducido del \(0\).
Para tratar este tipo de situaciones será útil lo que sigue.

\textbf{Definición: Punto de acumulación}

Sean \(A\) y \(\xi\) un subconjunto y un punto de \(\mathbb{R}\)
respectivamente. \(\xi\) es un \textbf{punto de acumulación} de \(A\),
si en todo entorno abierto de \(\xi\) hay puntos de \(A\) diferentes de
\(\xi\).

Obsérvese que la definición anterior también se puede hacer en términos
de los \textbf{entornos reducidos} de un punto, es decir del conjunto de
puntos del entorno diferentes del centro: Un punto \(\xi\) es de
acumulación del conjunto \(A\) si en todo entorno reducido del punto
\(\xi\) hay puntos de \(A\).

\end{frame}

\begin{frame}{Puntos de acumulación: \textbf{Ejemplos}}
\protect\hypertarget{puntos-de-acumulaciuxf3n-ejemplos}{}

\textbf{Ejemplos}

\textbf{Ejemplo 1.} Los extremos \(a\),\(b\) de un intervalo abierto
\((a,b)\), son puntos de acumulación del intervalo. Son ejemplos de
puntos de acumulación de un conjunto que no pertenecenc al conjunto.

\textbf{Ejemplo 2.} Sea \(A=\{c\} \cup [a,b]\), donde
\(c \in \mathbb{R} \setminus [a,b]\). Entonces \(c\) no es un punto de
acumulación de \(A\), en tanto que todos los puntos del intervalo
\([a,b]\) lo son. De hecho los puntos del intervalo cerrado \([a,b]\)
son los únicos puntos de acumulación del intervalo.

\textbf{Ejemplo 3.} El conjunto \(\mathbb{N}\) de los números naturales
no tiene puntos de acumulación: \(n\) es el único natural en el
intervalo abierto \((n-1,n+1)\).

\end{frame}

\begin{frame}{Puntos de acumulación: Caracterización}
\protect\hypertarget{puntos-de-acumulaciuxf3n-caracterizaciuxf3n}{}

\textbf{Proposición}

\(\xi \in \mathbb{R}\) es un punto de acumulación de \(A\) si, y sólo
si, en cada entorno de \(\xi\) hay infinitos puntos de \(A\).

\textbf{Demostración }

Supongamos que dado un \(\epsilon >0\), en el entorno de \(\xi\)
definido por \((\xi - \epsilon, \xi+\epsilon)\) hay sólo un número
finito de puntos de A: \(\{a_1, \ldots, a_k\}\) distintos de \(\xi\).
Sea ahora \(\delta = \inf\{ |a_i-\xi|: i=1,\ldots,k\}\). Es evidente que
si \(x\) es un punto del entorno abierto \((\xi-\delta,\xi+\delta)\),
entonces es \(|x-\xi|<\delta\), por lo que no puede ser uno de los
\(a_i\) y, por lo tanto, este seria un entorno de \(\xi\) sin puntos de
\(A\) diferentes de \(\xi\), lo que contradice que \(\xi\) sea un punto
de acumulación de \(A\).

El recíproco es inmediato.

\end{frame}

\begin{frame}{Teorema de Bolzano-Weierstrass}
\protect\hypertarget{teorema-de-bolzano-weierstrass}{}

\textbf{Proposición: Teorema de Bolzano-Weierstrass}

Todo conjunto de números reales infinito y acotado tiene un punto de
acumulación.

\textbf{Demostración}

Sea \(A\subset \mathbb{R}\) infinito y acotado, veremos que tiene un
punto de acumulación.

Por estar \(A\) acotado, es decir que existe \(K>0\) tal que \(|a|<K\)
para todo \(a \in A\), es \(A \subset [-K,K]\).

Por tener \(A\) infinitos puntos, en al menos uno de los intervalos
\([-K,0]\) y \([0,K]\) hay infinitos puntos de \(A\). Sea
\(I_1 =[a_1,b_1]\) esta mitad. (Si los hay en los dos, nos quedamos con
el de la izquierda, por ejemplo). Conviene observar que
\(|b_1-a_1|=\dfrac{K}{2}\).

Dado que en \(I_1 =[a_1,b_1]\) hay infinitos puntos de \(A\), podemos
repetir el proceso y determinar así un intervalo \(I_2 = [a_2,b_2]\),
que seria una de las dos mitades \([a_1, \frac{a_1+b_1}{2}]\) o
\([\frac{a_1+b_1}{2},b_1]\), con infinitos puntos de \(A\) y tal que
\(I_2 \subset I_1\) y que
\(b_2-a_2= \dfrac{b_1 - a_1}{2}= \dfrac{K}{2^2}\).

\end{frame}

\begin{frame}{Teorema de Bolzano-Weierstrass (cont.)}
\protect\hypertarget{teorema-de-bolzano-weierstrass-cont.}{}

\textbf{Demostración}

Reiterando este proceso, construiremos una sucesión \(I_n\) de
intervalos anidados \(I_n\), tal que \(I_{n+1} \subset I_n\) y
\(b_n-a_n = \dfrac{K}{2^n}\).

Además, el \(\inf \left\{\dfrac{k}{2^n}: n \in \mathbb{N} \right\}\),
existe puesto que se trata de un conjunto acotado inferiormente por
\(0\). Este ínfimo no puede ser positivo, puesto que si existe
\(0 < r <\dfrac{1}{2^n}\) para todo \(n \in \mathbb{N}\), resultaria que
tendria que ser \(2^n < \dfrac{1}{r}\), o lo que es lo mismo,
\(n \log 2 <-\log r\) y, por lo tanto, \(n < -\dfrac{\log r}{\log 2}\),
lo que significaría que los números naturales estan acotados en
\(\mathbb{R}\) y ya hemos visto que no puede ser.

Ahora, por el principio de los intervalos anidados, tendremos que hay un
solo punto \(\xi\) que está en todos los intervalos. \(\xi\) es el punto
de acumulación buscado, puesto que por ser \(\xi\) el infimo de los
\(b_n\) y el supremo de los \(a_n\), dado un \(\epsilon >0\), existe un
\(n \in \mathbb{N}\) tal que
\(\xi-\epsilon \leq a_n \leq \xi \leq b_n \leq \xi+ \epsilon\), y, por
lo tanto, cualquier entorno de \(\xi\) contiene infinitos puntos de
\(A\).

\end{frame}

\begin{frame}{Recta real ampliada}
\protect\hypertarget{recta-real-ampliada}{}

A menudo será conveniente considerar conjuntos como
\(\{ x \in \mathbb{R}: x \geq a \}\) o
\(\{x \in \mathbb{R}: x \leq a \}\), donde \(a\) indica un número real.
Para referirnos a estos conjuntos es útil introducir los símbolos más
infinito, \(+ \infty\), y menos infinito, \(- \infty\), tales que, para
todo número real \(x\) es \(-\infty < x < +\infty\). Siendo así, podemos
escribir \[
\{ x \in \mathbb{R}: x \geq a \} = [a,+\infty) \quad  \text{ y } \quad \{ x \in \mathbb{R}: x \leq a \} = (-\infty,a]
\] \textbf{Definición: Recta real ampliada.}

Llamamos \textbf{recta real ampliada}, \(\overline{\mathbb{R}}\) al
conjunto resultante de añadir los símbolos \(-\infty\) y \(+\infty\) al
conjunto de los números reales, es decir

\[
\overline{\mathbb{R}} = \mathbb{R} \cup \{ -\infty,+\infty \}
\]

\end{frame}

\begin{frame}{Operaciones aritméticas en la recta real ampliada}
\protect\hypertarget{operaciones-aritmuxe9ticas-en-la-recta-real-ampliada}{}

Obsérvese que, con la introducción de estos símbolos, podemos escribir:

\[
\mathbb{R}=(-\infty,+\infty) \quad \text{ y } \quad \overline{\mathbb{R}} = [-\infty,+\infty]
\] También es posible extender las operaciones aritméticas a
\(- \infty\) y \(+\infty\). Lo hacemos en la forma que indican las
tablas siguientes:

\begin{longtable}[]{@{}cc@{}}
\toprule
\endhead
\(x+(+\infty) = +\infty\) & \(x+(-\infty) = -\infty\)\tabularnewline
\(x-(+\infty) = -\infty\) & \(x-(-\infty) = +\infty\)\tabularnewline
\(x \cdot (+\infty) = +\infty\), si \(x>0\) &
\(x \cdot (+\infty) = -\infty\), si \(x<0\)\tabularnewline
\(x \cdot (-\infty) = -\infty\), si \(x>0\) &
\(x \cdot (-\infty) = +\infty\), si \(x<0\)\tabularnewline
\bottomrule
\end{longtable}

\end{frame}

\begin{frame}{Operaciones aritméticas en la recta real ampliada}
\protect\hypertarget{operaciones-aritmuxe9ticas-en-la-recta-real-ampliada-1}{}

\begin{longtable}[]{@{}cc@{}}
\toprule
\endhead
\(\dfrac{x}{+\infty} =0\) & \(\dfrac{x}{-\infty} =0\)\tabularnewline
\((+\infty) + (+\infty) = +\infty\) &
\((-\infty) + (-\infty)= -\infty\)\tabularnewline
\((+\infty) \cdot (+\infty) = +\infty\) &
\((+\infty) \cdot (-\infty) = -\infty\)\tabularnewline
\((-\infty) \cdot (-\infty) = +\infty\) &\tabularnewline
\bottomrule
\end{longtable}

\end{frame}

\begin{frame}{Indeterminaciones}
\protect\hypertarget{indeterminaciones}{}

Las operaciones siguientes tienen un resultado indeterminado: \[
\infty - \infty, \quad 0\cdot (\pm \infty), \quad 1^{\infty}, \quad \frac{0}{0}, \quad  \frac{\infty}{\infty}, \quad 0^0, \quad  \infty^0
\]

\end{frame}

\begin{frame}{Apartat/trans}
\protect\hypertarget{apartattrans}{}

\href{https://www.wolframalpha.com/input/?i=taylor+series+arctan\%281\%2F\%281\%2Bx\%5E6\%29\%29+at+x+\%3D+pi\%2F4}{\includegraphics{Images/wolfram.png}}

\href{https://www.wolframalpha.com/input/?i=6\%2F7+with+48+decimals}{\includegraphics{Images/wolfram.png}}

\end{frame}

\begin{frame}[fragile]{Apartat/trans2}
\protect\hypertarget{apartattrans2}{}

\begin{itemize}
\tightlist
\item
  això és un element
\item
  això és un altre
\end{itemize}

\[\int_0^1 x\ dx 4\]

\begin{verbatim}
## Expression : sin(x)/x
\end{verbatim}

\begin{verbatim}
## Limit of the expression tends to 0 : 1
\end{verbatim}

\end{frame}

\end{document}
